% Klassifiziert den Dokumenten-Typ
% Doku: http://exp1.fkp.physik.tu-darmstadt.de/tuddesign/
% Farben: http://www.tu-darmstadt.de/media/medien_stabsstelle_km/services/medien_cd/das_bild_der_tu_darmstadt.pdf
%  bigchapter: Chapter haben doppelte Schriftgröße
%  linedtoc: Linien im Inhaltsverzeichnis wie bei Überschriften
%  colorbacktitle: Der Dokumenten-Titel wird mir der Accentfarbe hinterlegt
\documentclass[bigchapter,colorback,accentcolor=tud4b,linedtoc,11pt]{tudreport}

% Input Dokument hat das Encoding UTF-8
\usepackage[utf8]{inputenc}
% Wichtiges Paket für Links und verlinktes Inhaltsverzeichnis
\usepackage[ngerman]{hyperref}
% Paket für Fußnoten
\usepackage[stable]{footmisc}
% Paket für Bibliotheks-Verzeichnis, square: Verwende eckige statt runde klammern
% \usepackage[square]{natbib}
% Paket zum Plotten von Datensätzen
\usepackage{pgfplots}
% Verwende deutsche Bezeichner für Inhaltsverzeichnis, ... (ngerman = New German: neue Rechtschreibung)
\usepackage{ngerman}
% Modul für chemische Formeln
\usepackage{chemformula}
% Deutsche Zahlen (entfernt z.B. das Leerzeichen nach einem Dezimal-Komma)
\usepackage{ziffer} 

\usepackage[verbose]{placeins}


% PDF-Optionen
\hypersetup{
  pdftitle={TU Darmstadt \- Physikalisches Praktikum für Fortgeschrittene},
  pdfauthor={Manuel Kress und Sören Link},
  pdfsubject={Versuch 4.9},
  pdfview=FitH,
}
% Nummeriere formeln in Subsections einzeln
\numberwithin{equation}{subsection}
% Kleines makro zur assymetrischen Fehlerangabe
\def\tol#1#2#3{\hbox{\rule{0pt}{15pt}${#1}^{+{#2}}_{-{#3}}$}}% 

%BEGINN TITELSEITE

\title{Laserresonator}

\subtitle{Manuel Kress  \\Sören Link}

\subsubtitle{Betreuer: Patric Ackermann \hfill Versuchsdatum: 10. Februar 2014}

\author{Manuel Kress, Sören Link}

\settitlepicture{img/title.jpg}

\institution{Physikalisches Praktikum \\für Fortgeschrittene \\ Versuch 4.9}

\date{\today}

%ENDE TITELSEITE

\begin{document}
%ANFANG DOKUMENT

%Titelseite einfügen
\maketitle

%Inhaltsverzeichnis einfügen
\tableofcontents

%ANFANG INHALT

\chapter{Einleitung}

\chapter{Grundlagen}
\section{Laserprinzip}
Die Grundvorraussetzung für jeden Laser ist die Besetzungsinversion. Dazu müssen sich für zwei Energieniveaus im Lasermedium, zwischen denen ein Übergang durch Emission von Photonen erlaubt ist, im energetischeren Niveau mehr Atome befinden als im weniger energetischen. Nur wenn diese Vorraussetzung erfüllt ist, können spontan emmitierte photonen mehr Emission als Absorption anregen, da beide vorgänge proportional zur Anzahl der Atome in den zugehörigen Energieniveaus sind.

Da die Einsteinkoeffizienten für stimulierte Emission und Absorption gleich groß sind ($B_{12}=B_{21}$), ist es unmöglich, ein 2-Niveau System zum lasern anzuregen, da unabhängig von der Stärke des Pumpvorgangs und unter Vernachlässigung der spontanten Emission maximal gleichbesetzung der Niveaus 1 und 2 erreicht werden kann.

Aus diesem Grund benutzt man für Laser zumeist 3- oder 4-Niveau Systeme. Dabei werden durch den Pumpvorgang Atome vom Grundniveau $E_0$ auf ein kurzlebiges niveau $E_1$ angeregt. Dieses geht sehr schnell durch spontane Emission in den langlebigen Zustand $E_{L1}$ über. 3- und 4-Niveau Systeme unterscheiden sich dadurch, dass im 3-Niveau System der Laservorgang von $E_{L1}$ nach $E_{0}$ stattfindet, wärend im 4-Niveau System ein kurzlebiger Zustand $E_{L2}$ zwischengeschaltet ist. Dies erlaubt das Betreiben von Lasern mit einem 4-Niveau System bei bereits sehr geringer Pumpleistung, da auch bei sehr geringer Anregung bereits Besetzungsinversion zwischen den relevanten Laser-Niveaus besteht.

Beim HeNe-Laser im speziellen werden Helium-Atome durch Gasentladungen von einem Grundzustand in die Zustände $2^{9}S_{1}$ und $2^{0}S_{0}$, welche beide metastabil sind, angeregt. Die so angeregten Heliumatome geben ihre Energie durch stöße an das Neon ab, welches dadurch in den 3s bzw 2s Zustand angeregt wird. Ermöglicht wird dieser Energietransfer durch den Geringen Energieunterschied der betroffenen Helium- und Neon Energieniveaus. Der Laserforgang selbst erfolg dann beim Übergang der Neon-Atome von dem 3s zum 2p Zustand, letzterer geht dann relativ schnell durch spontane Emission in den 1s Zustand über, welcher nach Stößen mit der Wand des Verstärkermediums in den Grundzustand übergeht.

\begin{figure}[ht!]
\centering
\includegraphics[width=65mm]{img/5074.png}
\caption{Energieschema eines HeNe-Lasers}
\label{HeNeLaser}
\cite{HeNeNiveaus}
\end{figure}
\section{Laserozillatoren}

\section{Resonatortheorie}

\section{Gefahren durch Laserstrahlung}

\chapter{Durchführung}
\section{Ausgangsleistung des Lasers in Abh. von der Resonatorlänge}
Zur Bestimmung der Stabilitätsgrenze des Resonaturs wurde die Ausgangsleistung für 11 Resonatorlängen aufgenommen, wobei eine Resonatorlänge innerhalb von 2 cm der errechneten Stabilitätsgrenze befand.
Die Spiegel des Laserresonators wurden dazu symmetrisch zum Verstärkermedium voneinander entfernt und die Ausgangsleistung des Resonators wurde mit dem S120C Messkopf, angeschlossen an PM100D Messgerät, aufgenommen. Anschließend wurde versucht, die aufgenommene Leistung durch Justage des Laserrohrs und der Resonatorspiegel zu optimieren.

Zu erwarten war hierbei eine annährend konstante Leistung des Lasers bis einige cm vor der Stabilitätsgrenze, mit einem kleinen Einbruch bei etwa der halben Stabilitätsgrenze.

Als Fehler für die jeweilige Spiegelposition wurden 1.5mm angenommen. Der Fehler für die aufgenommene Leistung wurde über die Fluktuation am Leistungsmessgerät abgeschätzt


\begin{center}
\begin{figure}[h]
\begin{tikzpicture}
\begin{axis}[
    title={Ausgangsleistung des Laserresonators in Abhängigkeit der Resonatorlänge},
    xlabel=Resonatorlänge in cm,
    ylabel=Ausgangsleistung in mW,
    width=0.9\textwidth,
    height= 11cm,
    xmin=20,
    xmax=92,
    grid=both,
    ymin=0,
    ymax=0.6,
    tick align=outside,
    tickpos=left,
    minor x tick num=3,
    minor y tick num=4,
    minor grid style={dotted,thin}
]
\addplot[only marks, mark=x, mark size=1pt, error bars/.cd, y dir=both, x dir=both]
table[x index={0},y index={2}, x error index ={1}, y error index={3}] {messdaten/ausgangsleistung.lvm};
\end{axis}
\end{tikzpicture}
\captionof{figure}{Ausgangsleistung des Laserresonators in Abhängigkeit der Resonatorlänge. Zu sehen ist die annährend konstante Leistugn des Resonators bis hin zu etwa 88cm und der Abfall der Leistung bei etwa der halben Stabilitätsgrenze und bei 89cm Resonatorlänge. Fehlerbalken sind eingezeichnet aber zu klein um sie zu sehen}
\end{figure}
\end{center}


\FloatBarrier
\section{Strahlbreite der Grundmode in Abh. von der Resonatorlänge}
Zur Bestimmung der Strahlbreite der Grundmode in Abhängigkeit von der Resonatorlänge wurde für 10 verschiedene Resonatorlängen ein Querschnitt durch die $TEM_{00}$ Mode mit einer CCD-Kamera aufgenommen. Um nur die Grundmode anzuregen, wurde das Laserrohr verkippt, bis alle anderen Moden durch Beugungsverluste eliminiert waren.

Hierbei ist zu erwarten, dass die Strahlbreite $w\left(\frac{L}{2}\right)$ mit zunehmender Resonatorlänge mit der Beziehung 
$$w\left(\frac{L}{2}\right)=w_0\cdot\sqrt{1+\frac{L\lambda}{2w_{0}^{2}\pi}}$$
zunimmt.

\FloatBarrier
\section{Longitudinale Modensruktur in Abh. von der Resonatorlänge}
Die Longitudinale Modenstruktur wurde bestimmt, indem für alle 10 Resonatorlängen die $TEM_{00}$ Mode durch eine Iris und auf auf ein Fabry-Perot Interferometer gelenkt wurde. Das Ausganssignal des Interferometers wurde dann mit einem Digitaloszilloskop verarbeitet.
Da die optische Achse des Laserresonators nicht perfekt kalibiriert war, war bei jeder Messung eine Nachjustierung der Position des Interferometers notwendig. Zwar gelang es uns für jede Resonatorlänge die Longitudinalen Moden mit dem Interferometer aufzunehmen, teilweise erscheinen die Lorenz-Peaks allerdings etwas verwaschen.

Für den Abstand der Longitudinalen Moden im Resonator ist folgender Zusammenhang zu erwarten:
$$\Delta\nu=\frac{c}{2L}$$
\FloatBarrier
\newpage
\section{Verstärkungsbandbreite des HeNe-Lasers}
Zur Bestimmung der Verstärkungsbandbreite des HeNe-Lasers wurde die Länge des Resonators auf 81cm eingestellt. Anschließend wurde das Laserrohr so justiert, dass nur noch die $TEM_{00}$ Mode angeregt wird. Das so angeregte Verstärkungsprofil wurde nun bei einer Ausgangsleistung der Mode von etwa 191mW mit Hilfe eines digitalen Oszilloskops über mehrere Sekunden hinweg aufgenommen. Anschließend wurde das Laserrohr so verkippt, dass auf Grund von Beugungsverlusten die Leistung auf etwa 101mW reduziert abfiel. Im Anschluss wurde erneut das Verstärkungsprofil der $TEM_{00}$ Mode aufgenommen.

\begin{figure}[ht!]
\centering
\includegraphics[width=80mm]{Messdaten/81cm191mW4uW.png}
\caption{Verstärkungsprofil des HeNe-Lasers bei einer Ausgangsleistung von etwa 191mW. Für bessere Sichtbarkeit des relevanten bereichs wurden teile der Achsen entfernt. Das Bild ist im Original im Anhang beigefügt}
\label{HeNeBreite191mW}
\end{figure}
\begin{figure}[ht!]
\centering
\includegraphics[width=80mm]{Messdaten/81cm101mW4uW.png}
\caption{Verstärkungsprofil des HeNe-Lasers bei einer Ausgangsleistung von etwa 101mW. Für bessere Sichtbarkeit des relevanten bereichs wurden teile der Achsen entfernt. Das Bild ist im Original im Anhang beigefügt}
\label{HeNeBreite101mW}
\end{figure}
\FloatBarrier
Während bei der zweiten Messung wie zu erwarten ein deutlicher Intensitätsabfall zu erkennen ist, scheint die Breite des Verstärkungsprofils annähend gleich zu sein.
\section{Beobachtung höherer transversaler Moden}
Zur Beobachtung höhrer Transversaler Moden wurded ein Drahtkreuz in den Laserresonator eingebracht, um die $TEM_{00}$-Mode zu unterdrücken. Dies erlaubte es, durch gezieltes verkippen des Resonators, höhere Moden anzuregen. Die so angeregten höheren Moden wurden wie in 3.2 mit einer CCD Kamera aufgenommen. Um Rauschen zu unterdrücken wurde dabei über etwa eine Sekunde an Messdaten gemittelt. Da höhere Moden im Gegensatz zut $TEM_{00}$ Mode nicht rotationssymmetrisch sind, mussten die so aufgenommenen Messdaten noch rotiert werden, um einen Modenverlauf symmetrisch zu Paralellen der x- und y-Achsen zu erreichen.

Für die Intensitätsverteilung der Moden ist folgender Zusammenhang zu erwarten
$$I_{mn}(x, y, z) = I_0 \left[ H_m \left( \frac{ \sqrt{2} x}{w} \right) \exp \left( \frac{-x^2}{w^2} \right) \right]^2 \left[ H_n \left( \frac{ \sqrt{2} y}{w} \right) \exp \left( \frac{-y^2}{w^2} \right) \right]^2$$
\cite{TransModeIntensity}
wobei $H_k$ das k-te Hermitische Polynom ist.

\chapter{Auswertung}
\section{Ausgangsleistung des Lasers in Abh. von der Resonatorlänge}

\section{Strahlbreite der Grundmode in Abh. von der Resonatorlänge}

\section{Longitudinale Modensruktur in Abh. von der Resonatorlänge}

\section{Verstärkungsbandbreite des HeNe-Lasers}

\section{Beobachtung höherer transversaler Moden}

\chapter{Fazit}


%ENDE INHALT

\cleardoublepage{}
% Eintrag fürs Inhaltsverzeichnis

\newpage

\begin{thebibliography}{100}
  \bibitem{mappe} Literaturmappe zum Versuch aus der physikalischen Bibliothek
  \bibitem{HeNeNiveaus} \verb|http://lp.uni-goettingen.de/get/image/5074|
  \bibitem{TransModeIntensity} \verb|http://en.wikipedia.org/w/index.php?title=Transverse_mode&oldid=579051975|
\end{thebibliography}

\cleardoublepage{}
% Eintrag fürs Inhaltsverzeichnis
% Abbildungsverzeichnis einfügen
\addcontentsline{toc}{chapter}{Abbildungsverzeichnis}
\listoffigures

\end{document}
